\subsection{Risk analysis}

The level of risk is calculated by multiplying these factors, probability and impact, with one another. After calculating the risk, a number between 1-16 indicates the risk level, where 1 is the lowest possible risk and 16 is the highest possible risk. To more easily see what is an actual risk, we have defined an acceptable risk level of 4. Everything with a risk level below 4 is considered acceptable risk.

\subsubsection{Identify and Analyze project risks}
\begin{tabular}{|l|l|l|l|}
\hline
	Description 							& Probability & Impact 	& Risk level \\
\hline Loss of source code 						& 1		& 4 		& \cellcolor{green} 4 \\
\hline Project exceeds budget 					& 2 		& 3		& \cellcolor{orange} 6 \\
\hline Sudden change of requirement specification 		& 3 		& 2 		& \cellcolor{orange} 6 \\
\hline Project exceeds timeframe 				& 2 		& 2 		& \cellcolor{green} 4 \\
\hline Project cancellation 						& 1 		& 4		& \cellcolor{green} 4 \\
\hline Loss of communication between dev. team and users 	& 4 		& 2 		& \cellcolor{red} 8 \\
\hline
\end{tabular}
Fig - Risk table

\subsubsection{Plan for managing the most critical risks}

\paragraph{Project exceeds budget}~\\
Risk can be reduced by performing a monthly review of expenses, and by establishing a financial budget for the development timeframe.

\paragraph{Sudden modification of requirement specification}~\\
By maintaining a good and active dialogue between product owner and development team, preferably once every other week, this risk can be reduced.
Additionally, a thorough meeting will be held between product owner and development team led by the scrum master before the first sprint starts, to make sure all requirements and requests have been presented and considered..

\paragraph{Loss of communication between development team and users}~\\
This risk can be reduced by actively utilizing the user group. As soon as new functionality is complete it should be published for testing by the user group, as to allow them to continuously provide constructive feedback on the latest implemented features.
