\subsection{Project organization}
\subsubsection{Resources and responsibilities}
The development-company is rather small, with only 7 employees. No previous projects require maintenance, and thus 100\% of the available resources are allocated to develop the product for FrostByte.
Part of the tasks differ significantly from one another, so to make development as effective as possible over a small group of employees, the workload is split into tasks and distributed over the team.


The development team is split into the following groups:\\
\begin{itemize}
\item One project owner / employer\\
\item One SCRUM master\\
\item Two developers working on backend\\
\item Two developers working on solutions for handheld devices\\
\item One developer working on design
\end{itemize}

All employees are experienced and competent within these fields, and workload can thus be delegated rather flexibly if required.

\subsubsection{Miscellaneous roles and employment}

The development team does not have enough resources to perform thorough testing of the system, and thus an external user group will be established which will be given access to the system whilst it’s still under development. This will provide the development team with enough useful feedback and a good overview of bugs and missing features in the platform.\\
This user group will consist of volunteering invidiuals who’ve shown an interest in the project through various communication channels. The group will be as diverse as possible, with people of different background and expertise (or lack thereof), which should lead to a large variety of different perspectives and thus, more varied feedback.
A new version should be rolled out once every 2 weeks into the testing environment.

Security is an important focus for FrostByte, and an external security audit-team will therefore be brought in to make sure all development is done with security as part of the development life cycle. This will also help improving and maintaining the trust-relation between the company and its users.

\subsubsection{Choice of development methodology}
\paragraph{Daily and weekly SCRUM meetings}~\\
Every day, the developers will be given an overview of the goals for the day. Problems and challenges can be discussed and space will be given to make adjustments if needed.~\\

Every second week, at the end of each sprint, the developers should engage in a combined sprint review and sprint retrospective meeting, where they shall present a short overview of which tasks have been completed and what problems and challenges have arisen since the beginning of the last sprint.

Because these meetings are common to all developers, time will be allocated at the end of the meeting to allow for a short discussion of the potential issues and challenges.

\paragraph{Meetings with product owner}~\\
Every other week, a meeting between the product owner and the development team will be held to ensure an active dialog. During this meeting it will be possible to suggest adjustments and changes of the product, and for the product owner to be given a brief report of the current development state.
